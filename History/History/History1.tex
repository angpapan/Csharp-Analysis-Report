% \begin{minipage}{\textwidth}
Η \textbf{C\#} είναι μία γλώσσα προγραμματισμού Η/Υ. Δημιουργήθηκε από την \textbf{Microsoft}. Το \textbf{2000} η Microsoft ανακοίνωσε την δημιουργία της πλατφόρμας .Net. Το\textbf{ 1999} , δηλαδή 1 χρόνο πριν την δημιουργία του .Net, συγκροτήθηκε μία ομάδα με επικεφαλή τον Άντερς Χάιλσμπεργκ (ο οποίος θεωρείτε ο πατέρας της C \#) με σκοπό την δημιουργία μιας καινούργιας γλώσσας με όνομα \textbf{Cool} (C-like Object Oriented Language). Μέχρι τον Ιούλιο του 2000 που ανακοινώθηκε απο την Microsoft η πλατφόρμα .Net η γλώσσα είχε μετονομαστεί σε C\# στην οποία αργότερα εισήχθησαν οι βιβλιοθήκες της \textbf{ASP.NET}.

Η C\# όταν δημιουργήθηκε έμοιαζε πάρα πολύ με την \textbf{Java}. Προγραμματιστής της Java ονόματι \textbf{Τζέιμς Γκόσλινγκ} σχολίασε για την C\# ότι "Eίναι ίδια με την Java απλά χωρίς αξιοπιστία, παραγωγικότητα και ασφάλεια " αν και ο ιδρυτής της Άντερς Χάιλσμπεργκ υποστήριξε οτι δεν είναι κλώνος της Java αλλά ότι είναι πολύ κοντά στην C++. Με την κυκλοφορία της C\# 2.0 τον Νοέμβριο 2005 η Java και η C\# άρχισαν να απομακρύνονται προγραμματιστικά με σημαντική διαφορά στην υλοποίηση των \textbf{γενικών αντικειμένων}.

Ο κύριος σκοπός της C\# είναι να προσφέρει ένα ασφαλές, απλό, αλλά ισχυρό περιβάλλον για την ανάπτυξη λογισμικού που εκτελείται κυρίως στο .NET Framework της Microsoft. Η C\# σχεδιάστηκε για να είναι εύχρηστη για τους προγραμματιστές που ήδη γνωρίζουν C++ ή Java, προσφέροντας μια σύγχρονη γλώσσα με πλούσιες δυνατότητες αποδοτικότητας και ασφάλειας.

Παρέχει εργαλεία για την ανάπτυξη εφαρμογών Windows, Web services, διαδικτυακές εφαρμογές μέσω ASP.NET, καθώς και για εφαρμογές για κινητά και cloud. Επιπρόσθετα, η C\# υποστηρίζει σύγχρονες προγραμματιστικές αρχές και παραδείγματα, όπως η ενθυλάκωση, η κληρονομικότητα και η πολυμορφία, ενώ παράλληλα διαθέτει αυστηρό έλεγχο τύπων και αυτόματη διαχείριση μνήμης για την αποφυγή σφαλμάτων, όπως διαρροές μνήμης και άλλα.
% \end{minipage}

\begin{table}[h]
\centering
\label{my-label}
\resizebox{1.1\textwidth}{0.12\textheight}{%
\begin{tabular}{|l|l|l|l|l|l|l|}
\hline
\multicolumn{1}{|c|}{Έκδοση} & \multicolumn{3}{c|}{Γλώσσα Προγραμματισμού} & Ημερομηνία & \.NET Framework & Visual Studio \\ \cline{2-4}
 & ECMA & ISO/IEC & Microsoft &  &  &  \\ \hline
C\# 1.0 & Δεκέμβριος 2002 & Απρίλιος 2003 & Ιανουάριος 2002 & Ιανουάριος 2002 & .NET Framework 1.0 & Visual Studio .NET 2002 \\ \hline
C\# 1.2 & Δεκέμβριος 2002 & Απρίλιος 2003 & Οκτώβρης 2003 & Απρίλιος 2003   & .NET Framework 1.1 & Visual Studio .NET 2003 \\ \hline
C\# 2.0 & Ιούνιος 2006 & Σεπτέμβριος 2006 & Σεπτέμβριος 2005 & Νοέμβριος 2005 & .NET Framework 2.0 & Visual Studio 2005 \\ \hline
C\# 3.0 & \multicolumn{2}{c|}{Κανένα} & Αύγουστος 2007 & Νοέμβριος 2007 & .NET Framework 2.0/3.0/3.5 & Visual Studio 2008/2010 \\ \hline
C\# 4.0  & \multicolumn{2}{c|}{Κανένα} & Απρίλιος 2010 & Απρίλιος 2010 & .NET Framework 4 & Visual Studio 2010 \\ \hline
C\# 5.0  & \multicolumn{2}{c|}{Κανένα} & Ιούνιος 2013 & Αύγουστος 2012 & .NET Framework 4.5 & Visual Studio 2012/2013 \\ \hline
C\# 6.0  & \multicolumn{2}{c|}{Κανένα} & \multicolumn{2}{c|}{Κατάσταση: Δεν έχει κυκλοφορήσει ακόμα} & .NET Framework 4.6 & Visual Studio 2015 \\ \hline
\end{tabular}%
}
\end{table}

Ο παραπάνω πίνακας δείχνει την εξέλιξη της\textbf{ C\#} κατά τα χρόνια καθώς και τις συμβατότητες με \textbf{.Net Framework} και \textbf{Visual Studio}.



\textit{Το ECMA και το ISO/IEC είναι δύο σημαντικοί διεθνείς οργανισμοί που συμβάλλουν στην τυποποίηση διάφορων τεχνολογικών και βιομηχανικών προδιαγραφών, συμπεριλαμβανομένων των γλωσσών προγραμματισμού.}
