Η πλατφόρμα .Net της Microsoft είναι ένα ολοκληρωμένο αρχιτεκτονικό πλαίσιο και σύνολο τεχνολογιών που επιτρέπει την ανάπτυξη και την εκτέλεση πολλών τύπων εφαρμογών. Η πλατφόρμα υποστηρίζει παραδοσιακές εφαρμογές desktop-webpage εώς και μοντέρνες εφαρμογές κινητών και cloud-based εφαρμογές. Το .Net παρουσιάστηκε από τη Microsoft τον Ιούνιο του 2000, με την πρώτη έκδοση να κυκλοφορεί το \textbf{2002}. Η πρωτοβουλία .NET ήταν μέρος μιας μεγαλύτερης αλλαγής στην ανάπτυξη λογισμικού, με έμφαση στα Web services και την ανάπτυξη εφαρμογών μέσω ενός ενοποιημένου περιβάλλοντος. Η πρώτη έκδοση του .NET Framework περιλάμβανε το\textbf{ Common Language Runtime (CLR)}, το οποίο επέτρεπε την εκτέλεση προγραμμάτων γραμμένων σε διάφορες γλώσσες, καθώς και μια βιβλιοθήκη κλάσεων που παρείχε απαραίτητες λειτουργίες. Το \textbf{2005} ανακοινώθηκε η έκδοση \textbf{2.0} που περιλάμβανε σημαντικές βελτιώσεις, όπως την υποστήριξη για τον γενικευμένο προγραμματισμό, καλύτερη υποστήριξη για web εφαρμογές μέσω του ASP.NET και νέες τεχνολογίες για τη δημιουργία πιο δυναμικών web interfaces. 

Το \textbf{2008} αναβαθμίστηκε σε \textbf{.Net Framework 3.5} με την υποστήριξη του \textbf{Language Integrated Query (LINQ)} το οποίο επιτρέπει στους προγραμματιστές να γράφουν ερωτήματα δεδομένων στον κώδικα τους με πιο φυσικό τρόπο. Επίσης, περιελάμβανε ενσωματώσεις για την υποστήριξη των πλατφορμών web, όπως \textbf{AJAX} και \textbf{Web Services}. 

Το \textbf{2010} η αναβάθμιση σε \textbf{.NET Framework 4} έφερε την υποστήριξη για πιο προχωρημένες δυνατότητες, όπως βελτιωμένη παράλληλη επεξεργασία και καλύτερη διαχείριση μνήμης. Προστέθηκαν επίσης νέες δυνατότητες για τον προγραμματισμό των Windows Applications, όπως η βιβλιοθήκη Task Parallel Library. 

Το \textbf{2012} η αναβάθμιση σε \textbf{.NET Framework 4.5} παρουσίασε σημαντικές βελτιώσεις στην \textbf{ασφάλεια}, την \textbf{απόδοση}, και την ανάπτυξη web εφαρμογών, με περαιτέρω υποστήριξη για \textbf{HTML5} και \textbf{CSS3} στο ASP.NET.

Το \textbf{2015} η αναβάθμιση σε \textbf{.NET Framework 4.6} περιλάμβανε σημαντικές βελτιώσεις  στην υποστήριξη της γλώσσας C Sharp και το Visual Basic.

Το \textbf{2016} η Microsoft έκανε μια σημαντική αλλαγή πορείας με την κυκλοφορήσει του \textbf{ .NET Core RC1,} η οποία είναι μία νέα, cross-platform έκδοση του .NET που στοχεύει στην ανάπτυξη cloud-based και κινητών εφαρμογών. Το .NET Core σχεδιάστηκε να λειτουργεί σε διάφορα λειτουργικά συστήματα όπως \textbf{Windows}, \textbf{Linux},\textbf{MAC} και \textbf{macOS}, προσφέροντας τη δυνατότητα ανάπτυξης εφαρμογών με μεγάλη ευελιξία και προσβασημότητα. (Μέχρι τότε το .NET λειτουργούσε σε Windows)
\\[5\baselineskip]

Τα επόμενα χρόνια η Microsoft εστίασε στην αναβάθμιση του \textbf{.Net Core} ώστε να γίνει πιο ελκυστικό από τους προγραμματιστές με περισσότερες βιβλιοθήκες, καλύτερη απόδοση και νέες δυνατότητες.


\begin{table}[h]
    \centering
    \resizebox{1.1\textwidth}{0.12\textheight}{%
    \begin{tabular}{|l|l|l|l|l|l|l|l|l|l|l|l|l|l|l|l}
        \hline
        \textbf{Version} & \textbf{Released} & \textbf{Edition} & \textbf{Published}\\ \hline
        .NET Core RC1 & November 2015 & First & March 2016\\ \hline
        .NET Core 1.0 & June 2016 &  & \\ \hline
        .NET Core 1.1 & november 2016 &  & \\ \hline
        .NET Core 1.0.4 and .NET Core 1.1.1 & March2017 & Second & March 2017\\ \hline
        .NET Core 2.0 & August 2017 &  & \\ \hline
        .NET Core for UWP in Windows 10 Fall Creator Update & Octomber 2017 & Third & November 2017 \\ \hline
        .NET Core 2.1 (LTS) & May 2018 &  & \\ \hline
        .NET Core 2.2 (Current) & December 2018 &  & \\ \hline
        .NET Core 3.0 (Current) & September 2019 & Fourth & October 2019 \\ \hline
        .NET Core 3.1 (LTS) & December 2019 &  & \\ \hline
        Blazor WebAssembly 3.2(Current) & May 2020 &  & \\ \hline
        .NET 5.0 (Current) & November 2020 & Fifth & November 2020 \\ \hline
        .NET 6.0 (LTS) & November 2021 & Sixth & November 2021 \\ \hline
        .NET 7.0 (Current) & November 2022 & Seventh & November 2022 \\ \hline
        .NET 8.0 (LTS) & November 2023 & Eighth & November 2023 \\ \hline
    \end{tabular}
    }
    \label{tab:my_label}
\end{table}

\textbf{Παρακάτω ενδεικτικά  μερικές Γλώσσες Προγραμματισμού που υποστηρίζονται από το .Net}

\begin{enumerate}
    \item C\#
    \item Visual Basic .NET (VB.NET)
    \item F\#
    \item C++/CLI
    \item IronPython 
    \item IronRuby
    \item Boo
    \item PowerShell
    \item ASP.NET
    \item JScript.NET
    \item Unity
    \item Nemerle 
\end{enumerate}






