\begin{enumerate}
    \item \textbf{Τύπος Δεδομένων}
    \begin{itemize}
        \item \textbf{Structures (Δομές)}: Είναι τιμές (value types). Δημιουργούνται στη στοίβα (stack) και αντιγράφονται όταν ανατίθενται σε νέες μεταβλητές.
        \item \textbf{Classes (Κλάσεις)}: Είναι αναφορές (reference types). Δημιουργούνται στον σωρό (heap) και οι αναθέσεις τους αντιγράφουν αναφορές, όχι τα δεδομένα.
    \end{itemize}

    \item \textbf{Κατασκευαστές}
    \begin{itemize}
        \item \textbf{Structures}: Δεν επιτρέπεται να έχουν έναν προεπιλεγμένο κατασκευαστή χωρίς παραμέτρους (default constructor). Ο χρήστης μπορεί να ορίσει κατασκευαστές με παραμέτρους.
        \item \textbf{Classes}: Μπορούν να έχουν έναν προεπιλεγμένο κατασκευαστή χωρίς παραμέτρους καθώς και άλλους κατασκευαστές.
    \end{itemize}

    \item \textbf{Κληρονομικότητα}
    \begin{itemize}
        \item \textbf{Structures}: Δεν υποστηρίζουν κληρονομικότητα (inheritance). Δεν μπορούν να κληρονομήσουν από άλλες δομές ή κλάσεις και δεν μπορούν να χρησιμοποιηθούν ως βάση για άλλες δομές ή κλάσεις.
        \item \textbf{Classes}: Υποστηρίζουν πλήρως την κληρονομικότητα. Μπορούν να κληρονομήσουν από άλλες κλάσεις και να χρησιμοποιηθούν ως βάση για άλλες κλάσεις.
    \end{itemize}

    \item \textbf{Προεπιλεγμένη Ανάθεση}
    \begin{itemize}
        \item \textbf{Structures}: Όλες οι ιδιότητες πρέπει να έχουν τιμές πριν από την χρήση τους, είτε μέσω του κατασκευαστή είτε άμεσα.
        \item \textbf{Classes}: Οι ιδιότητες μπορούν να αρχικοποιηθούν σε οποιοδήποτε σημείο.
    \end{itemize}

    \item \textbf{Διαχείριση Μνήμης}
    \begin{itemize}
        \item \textbf{Structures}: Χρησιμοποιούν τη στοίβα (stack), πράγμα που σημαίνει ότι η διαχείριση μνήμης είναι συνήθως πιο γρήγορη, αλλά υπάρχει περιορισμός στον χώρο.
        \item \textbf{Classes}: Χρησιμοποιούν τον σωρό (heap), παρέχοντας μεγαλύτερη ευελιξία στη διαχείριση μεγάλων αντικειμένων και διασυνδεδεμένων δεδομένων.
    \end{itemize}

    \item \textbf{Τροποποιητές}
    \begin{itemize}
        \item \textbf{Structures}: Δεν μπορούν να είναι \texttt{abstract}, \texttt{sealed}, ή \texttt{protected}.
        \item \textbf{Classes}: Μπορούν να είναι \texttt{abstract}, \texttt{sealed}, και να χρησιμοποιούν τους τροποποιητές πρόσβασης \texttt{protected} και \texttt{internal}.
    \end{itemize}
\end{enumerate}
 \newpage
Παράδειγμα Σύγκρισης: 

\begin{listing}[H]
    \begin{minted}[]{csharp}
        using System;

        struct Point
        {
          public int X { get; set; }
          public int Y { get; set; }
        
          public Point(int x, int y)
          {
              X = x;
              Y = y;
          }
        
          public void Display()
          {
              Console.WriteLine($"Point is at ({X}, {Y})");
          }
        }
        
        class Program
        {
          static void Main()
          {
            Point p1 = new Point(10, 20);
            Point p2 = p1;// Αντιγραφή του p1 στο p2

            p2.X = 30;// Τροποποίηση του p2

            // Εμφάνιση των τιμών των p1 και p2
            p1.Display();// Εμφανίζει "Point is at (10, 20)"
            p2.Display();// Εμφανίζει "Point is at (30, 20)"
          }
        }
    \end{minted}
\caption{Παράδειγμα αντιγράφου Δομής}
\label{flagExec}
\end{listing}

\begin{listing}[H]
    \begin{minted}[]{csharp}
        using System;

        class PointClass
        {
          public int X { get; set; }
          public int Y { get; set; }
        
          public PointClass(int x, int y)
          {
              X = x;
              Y = y;
          }
        
          public void Display()
          {
              Console.WriteLine($"Point is at ({X}, {Y})");
          }
        }
        
        class Program
        {
          static void Main()
          {
              PointClass p1=new PointClass(10, 20);
              PointClass p2=p1;// Αντιγραφή αναφοράς του p1 στο p2
        
              p2.X=30;// Τροποποίηση του p2
        
              // Εμφάνιση των τιμών των p1 και p2
              p1.Display(); // Εμφανίζει "Point is at (30, 20)"
              p2.Display(); // Εμφανίζει "Point is at (30, 20)"
          }
        }
    \end{minted}
\caption{Αντιπαράδειγμα Κλάσης}
\label{flagExec}
\end{listing}

Όπως φαίνεται και στο παράδειγμα υπάρχει μια βασική διαφορά όταν δημιουργούμαι αντίγραφο κλάσης και όταν δημιούργουμαι αντίγραφο δομής. Το αντίγραφω κλάσης αναφέρεται στην ίδια κλάση και γι αυτό βλέπουμε ότι οταν αλλάζουμε στο αντίγραφο την τιμή αλλάζει και στο κύριο.
Δεν ισχύει το ίδιο με το αντίγραφο της δομής.