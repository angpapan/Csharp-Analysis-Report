Το ακρωνύμιο LINQ προέρχεται από τα αρχικά των λέξεων Language Integrated Query. Αποτελεί μία γλώσσας συγγραφής ερωτημάτων ανεπτυγμένη από την Microsoft η οποία είναι πλήρως ενταγμένη στη C\# και προσφέρει εύκολη πρόσβαση σε συλλογές δεδομένων από διάφορες πηγές. Υποστηρίζονται ερωτήματα σε αντικείμενα της γλώσσας, βάσεις δεδομένων, αρχεία XML και άλλα.

Με αυτό τον τρόπο παρέχεται στους προγραμματιστές ένας ενιαίος τρόπος συγγραφής ερωτημάτων, απαλλάσσοντάς τους από την ανάγκη να γνωρίζουν σε βάθος πολλές διαφορετικές τεχνολογίες.

Η σύνταξη linq στη συνέχεια μεταφράζεται στη γλώσσα της αντίστοιχης τεχνολογίας προκειμένου να εκτελεστεί.

Παρέχονται δύο διαφορετικοί τρόποι συγγραφής των ερωτημάτων, είτε με εκφράσεις  query, είτε με εκφράσεις lambda. Οι δύο τρόποι σύνταξης δεν έχουν διαφορά στις επιδόσεις. Οι εκφράσεις query αποτελούν συντακτική διευκόλυνση και μεταφράζονται σε εκφράσεις lamda κατά την πρώτη φάση της μεταγλώττισης.

Στον παράδειγμα του κώδικα \ref{linqExec} εφαρμόζονται και οι δύο τρόποι σύνταξης προκειμένου να γίνουν δύο ερωτήματα σε έναν πίνακα αλφαριθμητικών τιμών. Παρατηρείται ότι στο πρώτο ερώτημα που εντοπίζει λέξεις με περισσότερα από 5 γράμματα και χρησιμοποιεί εκφράσεις query, είναι εμφανής η ομοιότητα με τη γλώσσα SQL, διευκολύνοντας προγραμματιστές με αντίστοιχο υπόβαθρο.  

\begin{listing}[htbp]
\begin{minted}[]{csharp}
tring[] strings = { "di", "uoa", "compilers", "linq" };

IEnumerable<string> lengthy = 
    from str in strings 
    where str.Length > 5 
    select str;

IEnumerable<string> startWithD = strings
    .Where(s => s.StartsWith("d"))
    .Select(s => s);

foreach (string l in lengthy) Console.WriteLine(l); // compilers
foreach (string l in startWithD) Console.WriteLine(l); // di
\end{minted}
\caption{Ερωτήματα Linq}
\label{linqExec}
\end{listing}