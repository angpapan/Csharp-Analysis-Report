Στη C\# παρέχεται η δυνατότητα να δηλωθούν μεταβλητές χωρίς να δηλωθεί ρητά και ο τύπος τους, όπως στο παράδειγμα του κώδικα \ref{implicitDeclaration}. Η implicit δήλωση μεταβλητών μπορεί να πραγματοποιηθεί με τρεις διαφορετικούς τρόπους, ανάλογα με τον βαθμό ελευθερίας.

Ο πρώτος τρόπος και πιο συνηθισμένος τρόπος είναι η χρήση της λέξης κλειδιού \codebox{var}. Με αυτό τον τρόπο συνεχίζει να εφαρμόζεται η στατική δήλωση των μεταβλητών. Ο μεταγλωττιστής της γλώσσας αποφασίζει τον καταλληλότερο τύπο για τη μεταβλητή σύμφωνα με την τιμή αρχικοποίησής της κατά τον χρόνο μεταγλώττισης, μην επιτρέποντας στη συνέχεια την αλλαγή του τύπου της μεταβλητής. Για το λόγο αυτό δεν είναι δυνατό να δηλωθεί μία μεταβλητή με τη χρήση της \codebox{var} χωρίς, ταυτόχρονα, να αρχικοποιηθεί.

Ο δεύτερος τρόπος είναι η χρήση της λέξης κλειδιού \codebox{dynamic}. Χρησιμοποιώντας της η μεταβλητή μπορεί να αλλάζει τύπο όσες φορές χρειάζεται, στα πρότυπα των scripting γλωσσών προγραμματισμού όπως η python και η javascript. Ο τύπος της μεταβλητής αποφασίζεται κατά τον χρόνο εκτέλεσης. Η δυνατότητα αυτή, όμως, έρχεται με ένα σχετικά μεγάλο κόστος εκτέλεσης.

Ο τρίτος τρόπος είναι η χρήση της λέξης \codebox{object}. Η γενική λέξη αντικειμένου της γλώσσας επιτρέπει στη μεταβλητή να πάρει οποιαδήποτε τιμή, δεδομένης της αντικειμενοστρέφειας της γλώσσας. Πρέπει, όμως, να σημειωθεί πως οι πιθανές μέθοδοι του τύπου της μεταβλητής δεν μπορούν να χρησιμοποιηθούν άμεσα, αλλά επιβάλλεται να γίνει casting. 

\begin{listing}[htbp]
\begin{minted}[]{csharp}
var staticVar = "compilers"; // string
staticVar = 10; // error

dynamic dynamicVar = "compilers"; // runtime string
Console.WriteLine(dynamicVar.Length); // 9
dynamicVar = 10; // ok

object objectVar = "compilers";
Console.WriteLine(objectVar.Length); // error
objectVar = 10; // ok
\end{minted}
\caption{Implicit δήλωση μεταβλητών}
\label{implicitDeclaration}
\end{listing}