Το exception handling (χειρισμός εξαιρέσεων). είναι ένα κρίσιμο κομμάτι της προγραμματιστικής διαδικασίας, για τη συγγραφή στιβαρού και αξιόπιστου κώδικα. Επιτρέπει στους προγραμματιστές να χειρίζονται σφάλματα και εξαιρέσεις που μπορεί να προκύψουν κατά τη διάρκεια εκτέλεσης του προγράμματος. Έτσι οι εφαρμογές δεν τερματίζουν απροσδόκητα, ενισχύοντας την σταθερότητά τους. Βοηθάει στον εντοπισμό και τη διάγνωση σφαλμάτων (debugging), παρέχοντας ουσιαστικά μηνύματα. Ακόμη, επιστρέπει στην εφαρμογή να ανακάμπτει από απρόσμενες συνθήκες, συνεχίζοντας την εκτέλεση ή κάνοντας διορθωτικές ενέργειες.

Στη C\# παρέχεται ένας δομημένος μηχανισμός για τον χειρισμό των εξαιρέσεων, χρησιμοποιώντας τα μπλοκ try-catch. Το κομμάτι του κώδικα που ενδέχεται να προκαλέσει κάποια εξαίρεση περικλείεται σε ένα μπλοκ \codebox{try}. Εάν προκύψει εξαίρεση, τότε αυτή "πιάνεται" και αντιμετωπίζεται μέσα σε ένα μπλοκ \codebox{catch}. Με αυτό τον τρόπο αποτρέπεται ο απότομος τερματισμός του προγράμματος. Τέλος, υπάρχει το μπλοκ \codebox{finally}, που είναι προαιρετικό και εκτελείται ανεξάρτητα από το αν συμβεί μια εξαίρεση. 

Η C\# παρέχει μια ποικιλία ενσωματωμένων τύπων εξαιρέσεων, όπως οι \linebreak \textbf{DivideByZeroException, FileNotFoundException, ArgumentNullException} και άλλοι, που αντιπροσωπεύουν κοινά σενάρια που μπορεί να συναντήσει ένας προγραμματιστής στον κώδικά του. Στο παράδειγμα του κώδικα \ref{exception_handling}, η \linebreak \textbf{DivideByZeroException} "πιάνεται" κατά τη προσπάθεια διαίρεσης ενός αριθμού (ακεραίου ή δεκαδικού) με το μηδέν. Ο προγραμματιστής μπορεί να χειριστεί αυτό το σενάριο όπως κρίνει σκόπιμο (π.χ. εμφάνιση ενός μηνύματος). 

\begin{listing}[htbp]
\begin{minted}[]{csharp}
try
{
    // Code that may throw an exception
    int a = 100;
    int b = 0;
    // Division by zero, throws DivideByZeroException
    int result = a / b;  
}
catch (DivideByZeroException ex)
{
    // Prints the message "Attempted to divide by zero."
    Console.WriteLine(ex.Message);
}
catch (Exception ex)
{
    // Handle other types of exceptions
    Console.WriteLine("An error occurred: " + ex.Message);
}
finally
{
    // Optional block for cleanup code (runs every time)
    Console.WriteLine("Cleanup code");
}
\end{minted}
\caption{Χειρισμός διαίρεσης με το μηδέν}
\label{exception_handling}
\end{listing}