Η χρήση των Generics επιτρέπει στην ίδια κλάση ή μέθοδο να χρησιμοποιείται με διαφορετικούς τύπους δεδομένων, διευκολύνοντας την επαναχρησιμοποίηση του κώδικα.

Για τη δήλωση μίας generic κλάσης χρησιμοποιούνται τα \codebox{<>} αμέσως μετά το όνομά της, όπως στο παράδειγμα του κώδικα \ref{genericsPersonClass}. Το \codebox{T} ανάμεσα στα \codebox{<>} ονομάζεται μεταβλητή τύπου. Η κλάση του κώδικα υποστηρίζει για το κάθε άτομο ένα Id οποιουδήποτε τύπου, πέρα από το όνομα και τον τίτλο του. Ο τύπος του Id ορίζεται κατά τη δήλωση ενός αντικειμένου και δεν μπορεί να αλλάξει κατά τον χρόνο εκτέλεσης. Σε περίπτωση που δοθεί διαφορετικός τύπος δεδομένων από αυτόν που ορίστηκε προκύπτει σφάλμα κατά τον χρόνο μεταγλώττισης του προγράμματος. 

\begin{listing}[htbp]
\begin{minted}[]{csharp}
internal class Person<T>
{
    private T _id;
    public T Id => _id;
    public string Name { get; set; }
    public string Title { get; set; }

    public Person(T id) { _id = id; }
    
    public Person(T id, string name, string title) : this(id)
    {
        Name = name; Title = title;
    }

    public override string ToString()
    {
        return $"Id: {_id}, Title: {Title}, Name: {Name}";
    }
}
\end{minted}
\caption{Κλάση με γενικό όρισμα}
\label{genericsPersonClass}
\end{listing}

Η χρήση μίας κλάσης που χρησιμοποιεί generics, γίνεται όπως στον κώδικα \ref{genericsExec}, όπου σε κάθε \codebox{Person} δίνεται διαφορετικός τύπος Id.

\begin{listing}[htbp]
\begin{minted}[]{csharp}
Person<int> galdalf = new Person<int>
    (1, "Gandalf", "The White");
Person<string> thranduil = new Person<string>
    ("THRNDL", "Thranduil", "King of the Elves");
Person<double> aragorn = new Person<double>(100.12);
aragorn.Name = "Aragorn";
aragorn.Title = "King of Gondor";

Console.WriteLine(galdalf);
Console.WriteLine(thranduil);
Console.WriteLine(aragorn);
\end{minted}
\caption{Χρήση κλάσης με γενικό όρισμα}
\label{genericsExec}
\end{listing}

Αντίστοιχα με τις κλάσεις, γενικά ορίσματα μπορούν να δέχονται και οι συναρτήσεις, όπως στον κώδικα \ref{genericsMethod}.

\begin{listing}[htbp]
\begin{minted}[]{csharp}
public void Print<T>(T data)
{
    Console.WriteLine(T);
}
\end{minted}
\caption{Συνάρτηση με generic όρισμα.}
\label{genericsMethod}
\end{listing}


