Τα delegates στη C\# είναι παρόμοια με τους δείκτες σε συναρτήσεις σε γλώσσες όπως η C και η C++.  Η δήλωσή τους αποτελεί μία "υπογραφή" για τις συναρτήσεις στις οποίες μπορεί να διατηρεί κάποια αναφορά. Η σύνταξη για τη δήλωση ενός delegate είναι η παρακάτω:

\codebox{delegate <return type> <delegate-name> <parameter list>}

Η δημιουργία ενός στιγμιότυπου από delegate γίνεται με τον ίδιο τρόπου που δημιουργούνται στιγμιότυπα κλάσεων, με τη χρήση του \codebox{new}.

Στο παράδειγμα του κώδικα \ref{multicast} και \ref{multicastExec} παρουσιάζεται ένα delegate \codebox{Calculation} το οποίο αποτελεί το πρότυπο συναρτήσεων οι οποίες δέχονται ως όρισμα έναν ακέραιο και δεν επιστρέφουν τίποτα. Η κλάση \codebox{Multicast}, η οποία διατηρεί έναν αριθμό στον οποίο εκτελεί πράξεις μέσω συναρτήσεων, δίνει στον χρήστη της τη δυνατότητα να χρησιμοποιήσει το delegate \codebox{Calculation}, μέσω του πεδίου \codebox{calc}. Εδώ, να σημειωθεί ότι υπάρχει η δυνατότητα συνδυασμού συναρτήσεων σε ένα delegate, οι οποίες εκτελούνται σειριακά. Έτσι, στο παράδειγμα χρήσης της κλάσης αρχικά δημιουργούνται στιγμιότυπα των μεθόδων \codebox{Add} και \codebox{Multiply} ως \codebox{Calculation}, τα οποία στη συνέχεια προστίθενται στο πεδίο \codebox{calc}, το οποίο και εν τέλει εκτελείται.

\begin{listing}[htbp]
\begin{minted}[]{csharp}
delegate void Calculation(int oper);

internal class Multicast
{
    int _num;
    public int Num => _num;
    public Calculation Calc { get; set; }
    public Multicast(int num) { _num = num; }
    public void Add(int oper) => _num += oper;
    public void Multiply(int oper) => _num *= oper;
}
\end{minted}
\caption{Κλάση που μπορεί να χρησιμοποιήσει ένα ή περισσότερα delegates}
\label{multicast}
\end{listing}
\begin{listing}[htbp]
\begin{minted}[]{csharp}
Multicast num = new(5);
Calculation c1 = new Calculation(num.Add);
Calculation c2 = new Calculation(num.Multiply);
num.Calc = c1;
num.Calc += c2;
num.Calc(10); // (5 + 10) * 10

Console.WriteLine(num.Num); // 150
\end{minted}
\caption{Χρήση delegates}
\label{multicastExec}
\end{listing}

Η βασική, όμως, χρήση των delegates είναι η δυνατότητα να δίνονται συναρτήσεις ως όρισμα, προκειμένου να εκτελούνται ως call-backs ή events. Με τη χρήση τους, μία συνάρτηση μπορεί να δέχεται ως όρισμα μία άλλη συνάρτηση την οποία να εκτελεί, ενισχύοντας την ευελιξία και την επαναχρησιμοποίηση του κώδικα. Σύχνα, συνδυάζεται με τη χρήση εκφράσεων lambda.

Για παράδειγμα, στον κώδικα \ref{delegTeam} η μέθοδος \codebox{PrintPlayers} της κλάσης \codebox{Team}, δέχεται ως όρισμα ένα delegate (με τη μορφή \codebox{Predicate}, οποίο θα συζητηθεί στη συνέχεια), το οποίο αξιοποιείται προκειμένου να τυπώσει μόνο τους παίχτες της που ικανοποιούν κάποιο κριτήριο το οποίο δίνεται κατά την κλήση της μεθόδου.

Η κλάση \ref{delegPlayer} αναπαριστά έναν παίχτη και η κλάση \ref{delegDataFactory} απλώς δημιουργεί δεδομένα από τις κλάσεις \codebox{Team} και \codebox{Player}. Ο κώδικας \ref{funcDelegExec} χρησιμοποιεί τις παραπάνω κλάσεις, τυώνοντας αντίστοιχα αποτελέσματα. Επιπλέον, ορίζει το delegate \codebox{TeamDelegate} το οποίο δέχεται ως όρισμα μία ομάδα και επιστρέψει ένα λογικό αποτέλεσμά. Όλες οι συναρτήσεις που δέχονται ως όρισμα delegate καλούνται με τη χρήση lambda expressions, αλλά θα μπορούσαν αντίστοιχα να δίνονται ως ορίσματα συναρτήσεις που ικανοποιούν το αντίστοιχο delegate.

\begin{listing}[htbp]
\begin{minted}[]{csharp}
internal class Team
{
    public string Name { get; init; }
    public List<Player> Players { get; set; } = new();
    public Team(string name) { Name = name; }
    public void PrintPlayers(Predicate<Player> criterion)
    {
        foreach(Player p in Players)
            if(criterion(p)) Console.WriteLine(p);
    }
}
\end{minted}
\caption{Κλάση αναπαράστασης μίας ομάδας}
\label{delegTeam}
\end{listing}
\begin{listing}[htbp]
\begin{minted}[]{csharp}
internal class Player
{
    public string Name { get; init; }
    public int HeightInCm { get; init; }
    public Team Team { get; set; }
    public Player(string name, int heightInCm, Team team)
    {
        Name = name;
        HeightInCm = heightInCm;
        Team = team;
    }

    public override string ToString() =>
        $"Name: {Name}, Height: {HeightInCm}";
}
\end{minted}
\caption{Κλάση αναπαράστασης ενός παίχτη}
\label{delegPlayer}
\end{listing}
\begin{listing}[htbp]
\begin{minted}[]{csharp}
internal static class DataFactory
{
    public static List<Team> CreateTeams()
    {
        List<Team> teams = new();
        teams.Add(new Team("Olympiakos"));
        teams.Add(new Team("Panathinaikos"));
        return teams;
    }

    public static List<Player> CreatePlayers(List<Team> teams)
    {
        List<Player> players = new();

        players.Add(new Player("Player 00", 193, teams[0]));
        players.Add(new Player("Player 01", 211, teams[0]));
        players.Add(new Player("Player 02", 201, teams[0]));

        players.Add(new Player("Player 10", 189, teams[1]));
        players.Add(new Player("Player 11", 209, teams[1]));
        players.Add(new Player("Player 12", 202, teams[1]));

        return players;
    }
}
\end{minted}
\caption{Κλάση δημιουργίας δεδομένων}
\label{delegDataFactory}
\end{listing}
\begin{listing}[htbp]
\begin{minted}[]{csharp}
delegate bool TeamDelegate(Team team); // == Predicate<Team>

void PrintPlayersByTeamCriterion(TeamDelegate criterion)
{
    foreach (Player player in players)
        if (criterion(player.Team)) Console.WriteLine(player);
}

List<Team> teams = DataFactory.CreateTeams();
List<Player> players = DataFactory.CreatePlayers(teams);
players.ForEach(p => p.Team.Players.Add(p));

Console.WriteLine($"{teams[1].Name} players over 205cm");
teams[1].PrintPlayers(p => p.HeightInCm > 205);
Console.WriteLine();

Console.WriteLine($"{teams[1].Name} players under 203cm");
teams[1].PrintPlayers(p => p.HeightInCm < 203);
Console.WriteLine();

Console.WriteLine($"Players playing for Olympiakos");
PrintPlayersByTeamCriterion(t => t.Name == "Olympiakos");
\end{minted}
\caption{Χρήση delegates}
\label{funcDelegExec}
\end{listing}

Στην πράξη, στην πλειονότητα των περιπτώσεων δεν ορίζονται νέα delegates από τους προγραμματιστές, αλλά χρησιμοποιούνται τα built-in delegates της γλώσσας. Αυτά είναι:

\begin{itemize}
\item \textbf{Action}: Αναπαριστά ένα delegate το οποίο δέχεται κανένα, ένα ή περισσότερα (έως 16) ορίσματα και δεν επιστρέφει τίποτα. Ορίζεται ως 
\newline 
\codebox{public delegate void Action<T>(T arg);} 
\newline 
και συντάσσεται ως 
\newline 
\codebox{Action<T1, T2, ...>}.

\item \textbf{Func}: Αναπαριστά ένα delegate το οποίο δέχεται κανένα, ένα ή περισσότερα (έως 16) ορίσματα και επιστρέφει κάποια τιμή. Ορίζεται ως 
\newline 
\codebox{public delegate Tout Func<Tin1, Tin2, Tout>(Tin1 arg1, Tin2 arg2);} 
\newline 
και συντάσσεται ως 
\newline 
\codebox{Func<Tin1, Tin2, ..., Tout>}.

\item \textbf{Predicate}: Αναπαριστά ένα delegate το οποίο δέχεται κανένα, ένα ή περισσότερα (έως 16) ορίσματα και επιστρέφει μία λογική τιμή (true/false). Ορίζεται ως 
\newline 
\codebox{public delegate bool Prdecate<Tin1, Tin2>(Tin1 arg1, Tin2 arg2);} 
\newline 
και συντάσσεται ως 
\newline 
\codebox{Predicate<Tin1, Tin2, ...>}.

\end{itemize}
