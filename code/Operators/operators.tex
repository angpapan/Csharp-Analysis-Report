Τελεστές C\#
Οι τελεστές είναι σύμβολα που χρησιμοποιούνται για την εκτέλεση πράξεων σε τελεστέους. Οι τελεστέοι μπορεί να είναι μεταβλητές και/ή σταθερές.

Για παράδειγμα, στο $2+3$, το + είναι ένας τελεστής που χρησιμοποιείται για την εκτέλεση της πράξη της πρόσθεσης ενώ οι αριθμοί 2,3 είναι οι τελεστέοι.

Οι τελεστές χρησιμοποιούνται για τη διαχείριση μεταβλητών και τιμών σε ένα πρόγραμμα. Η C\# υποστηρίζει έναν αριθμό τελεστών που ταξινομούνται βάσει του τύπου των πράξεων που εκτελούν.
\newpage 
\textbf{1. Βασικός Τελεστής Ανάθεσης (Basic Assignment Operator)}

\begin{listing}[H]
\begin{minted}[]{csharp}
// Defining a simple class
class AssignmentOperator
{
    public static void Main(string[] args)
    {
        int firstNumber, secondNumber;
        // Assigning a constant to variable
        firstNumber = 10;
        Console.WriteLine("First Number = {0}", firstNumber);

        // Assigning a variable to another variable
        secondNumber = firstNumber;
        Console.WriteLine("Second Number = {0}", secondNumber);
    }
}
\end{minted}
\caption{Ο τελεστής =}
\label{flagExec}
\end{listing}

\begin{verbatim}
    Output:
    First Number = 10
    Second Number = 10
\end{verbatim}
 Υπάρχουν πολλών ειδών τελεστές όπως θα δούμε παρακάτω με διαφορετικές λειτουργιές. Οι τελεστές είναι βασικοί πυλώνες δόμησης ενός προγράμματος και για αυτό τον λόγο είναι πολύ σημαντικό
 να κατανοηθεί η λειτουργία τους. Παρουσιάζονται οι τελεστές με ιεραρχία σημαντικότητας.
\newpage
\textbf{2. Αριθμητικοί Τελεστές (Arithmetic Operators)}

\begin{table}[h!]
    \begin{tabular}{|c|l|p{8cm}|}
    \hline
    \textbf{Operator} & \textbf{Name} & \textbf{Description} \\ \hline
    + & Addition & Adds two operands. \\ \hline
    - & Subtraction & Subtracts the second operand from the first. \\ \hline
    * & Multiplication & Multiplies two operands. \\ \hline
    / & Division & Divides the first operand by the second. \\ \hline
    \% & Modulus & Returns the remainder of a division operation. \\ \hline
    \end{tabular}
    \caption{Arithmetic Operators}
    \label{table:operators}
\end{table}


\textbf{3. Τελεστές Ανάθεσης (Assignment Operators)}

\begin{table}[h!]
    \begin{tabular}{|c|l|p{5cm}|}
    \hline
     \textbf{Operator} & \textbf{Name} & \textbf{Description} \\ \hline
     += & Addition Assignment & Adds the right operand to the left operand and assigns the result to the left operand. \\ \hline
     -= & Subtraction Assignment & Subtracts the right operand from the left operand and assigns the result to the left operand. \\ \hline
     *= & Multiplication Assignment & Multiplies the left operand by the right operand and assigns the result to the left operand. \\ \hline
     /= & Division Assignment & Divides the left operand by the right operand and assigns the result to the left operand. \\ \hline
     \%= & Modulus Assignment & Takes the modulus using two operands and assigns the result to the left operand. \\ \hline
    \end{tabular}
    \label{table:operators}
\end{table}

\textbf{4. Τελεστές Αύξησης/Μείωσης (Increment/Decrement Operators)}

\begin{table}[h!]
    \begin{tabular}{|c|l|p{8cm}|}
    \hline
     \textbf{Operator} & \textbf{Name} & \textbf{Description} \\ \hline
     ++ & Increment & Increases an integer value by one. \\ \hline
     -- & Decrement & Decreases an integer value by one. \\ \hline
    \end{tabular}
    \label{table:operators}
\end{table}

\newpage

\textbf{5. Τελεστές Άρνησης (Unary Operators)}

\begin{table}[h!]
    \begin{tabular}{|c|l|p{8cm}|}
    \hline
     \textbf{Operator} & \textbf{Name} & \textbf{Description} \\ \hline
     + & Unary Plus & Indicates a positive value. \\ \hline
     - & Unary Minus & Indicates a negative value. \\ \hline
    \end{tabular}
    \label{table:operators}
\end{table}

\textbf{6. Λογικοί Τελεστές (Logical Operators)}

\begin{table}[h!]
    \begin{tabular}{|c|l|p{8cm}|}
    \hline
    \textbf{Operator} & \textbf{Name} & \textbf{Description} \\ \hline
    \&\& & Logical AND & Returns true if both expressions are true. \\ \hline
    || & Logical OR & Returns true if one of the expressions is true. \\ \hline
    ! & Logical NOT & Reverses the logical state of its operand. \\ \hline
    \end{tabular}
    \label{table:operators}
\end{table}


\textbf{7. Τελεστές Bitwise (Bitwise Operators)}

\begin{table}[h!]
    \begin{tabular}{|c|l|p{5cm}|}
    \hline
    \textbf{Operator} & \textbf{Name} & \textbf{Description} \\ \hline
    \& & Bitwise AND & Performs a bitwise AND operation. \\ \hline
    | & Bitwise OR & Performs a bitwise OR operation. \\ \hline
    \textasciicircum & Bitwise XOR & Performs a bitwise XOR operation. \\ \hline
    \thicksim  & Bitwise Complement & Inverts all the bits of the operand. \\ \hline
    << & Left Shift & Shifts bits to the left. \\ \hline
    >> & Right Shift & Shifts bits to the right. \\ \hline
    \end{tabular}
    \label{table:operators}
\end{table}

Παραδείγματα χρήσης των παραπάνω τελεστών υπάρχουν στο \href{https://www.programiz.com/csharp-programming/operators}{Programiz}
