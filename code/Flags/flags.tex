Το χαρακτηριστικό (attribute) \codebox{[Flags]} χρησιμοποιείται σε enumerables το οποία αναπαριστούν ένα πλήθος πιθανών τιμών αντί για μία συγκεκριμένη. Τέτοια σύνολα χρησιμοποιούνται συχνά με bitwise τελεστές για την επιλογή περισσότερων ή λιγότερων επιλογών και τον έλεγχό τους. Οι τιμές τέτοιων enumerations πρέπει να είναι πάντα δυνάμεις του 2. Ένα τέτοιο enumeration αναπαρίσταται στον κώδικα \ref{flagsEnum}, το οποία περιγράφει τα πιθανά δικαιώματα που μπορεί να έχει ένας χρήστης, για παράδειγμα πάνω σε ένα αρχείο.

\begin{listing}[htbp]
\begin{minted}[]{csharp}
[Flags]
internal enum Permission
{
    None = 0,
    Execute = 1,
    Write = 1 << 1,
    Read = 1 << 2
}
\end{minted}
\caption{Flags Enum}
\label{flagsEnum}
\end{listing}

Έτσι, μπορούν να δίνονται αθροιστικά δικαιώματα σε έναν χρήστη όπως στο παράδειγμα του κώδικα \ref{flagsMainClass} και \ref{flagSubclasses}, όπου ένας administrator έχει πλήρη δικαιώματα, ενώ ένας χρήστης τύπου reader έχει μόνο δικαιώματα ανάγνωσης. Σημειώνεται ότι για την απόδοση περισσότερων δικαιωμάτων χρησιμοποιείται ο bitwise τελεστής \codebox{|}.

Αντίστοιχα, στον κώδικα \ref{flagFile} παρουσιάζεται μία κλάση που αναπαριστά ένα αρχείο το οποίο μπορεί είτε να διαβαστεί ή να γίνει εγγραφή δεδομένων σε αυτό. Κατά την κλήση αυτών των λειτουργιών γίνεται έλεγχος για τον αν ο χρήστης που τις καλεί έχει τα κατάλληλα δικαιώματα. Ο έλεγχος της ύπαρξης του αντίστοιχου δικαιώματος γίνεται είτε με τον bitwise τελεστή \codebox{&} είτε με κλήση της συνάρτησης \codebox{HasFlag(...)}. Οι δύο τρόποι είναι ισοδύναμοι, με τον bitwise τελεστή να υπερέχει ελάχιστα σε απόδοση.

\begin{listing}[htbp]
\begin{minted}[]{csharp}
internal abstract class Account
{
    public Permission Permissions { get; }
    public string UserName { get; }

    protected Account(Permission permissions, string userName)
    {
        Permissions = permissions;
        UserName = userName;
    }

    public void ListPermissions()
    {
        Console.WriteLine($"{UserName} has permission to: " +
            $"${Permissions}");
    }
}
\end{minted}
\caption{Κλάση για τη δημιουργία λογαριασμού χρήστη}
\label{flagsMainClass}
\end{listing}
\begin{listing}[H]
\begin{minted}[]{csharp}
internal class Admin : Account
{
    public Admin(string username) : 
        base(Permission.Read | Permission.Write 
            | Permission.Execute, username) { }
}

internal class Reader : Account
{
    public Reader(string username) : 
        base(Permission.Read, username) { }
}
\end{minted}
\caption{Admin και Reader accounts}
\label{flagSubclasses}
\end{listing}
\input{code/Flags/Code/flagFile}

Σημειώνεται ότι το χαρακτηριστικό \codebox{[Flags]} στην πραγματικότητα δεν είναι αυτό που επιτρέπει αυτές τις δυνατότητες, οι οποίες είναι διαθέσιμες ούτως ή άλλως. Το μόνο που προσφέρει πρακτικά είναι μία καλύτερη αναπαράσταση της μεθόδου \codebox{ToString()}, επιτρέποντας καλύτερη αναπαράσταση του enumeration. Η χρήση του παρουσιάζεται στη μέθοδο \codebox{ListPermissions()} της κλάσης του κώδικα \ref{flagsMainClass}. Είναι σημαντικό, όμως, ότι κάνει ξεκάθαρο πως το συγκεκριμένο enumeration αποτελεί συλλογή επιλογών και όχι απλώς αναπαράσταση κάποιας τιμής. Επιπλέον η χρήση του χαρακτηριστικού δεν επιβάλει από μόνη της την υλοποίηση των τιμών ως δυνάμεις του 2, κάτι για το οποίο είναι υπεύθυνος ο προγραμματιστής.

Τέλος, δίνεται ένα παράδειγμα χρήσης των παραπάνω κλάσεων στον κώδικα \ref{flagExec}.

\begin{listing}[H]
\begin{minted}[]{csharp}
Admin admin = new Admin("Admin");
Reader reader = new Reader("Reader");

admin.ListPermissions();
reader.ListPermissions();
Console.WriteLine();

Flags.File file = new Flags.File("test.txt");

file.Read(admin);
file.Read(reader);
file.Write(admin);
file.Write(reader);
\end{minted}
\caption{Χρήση κλάσης που χρησιμοποιεί Flag Enum}
\label{flagExec}
\end{listing}
