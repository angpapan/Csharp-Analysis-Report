Συχνά χρειάζεται να προστεθεί λειτουργικότητα σε υπάρχουσες κλάσεις, αλλά αυτό μπορεί να μην είναι δυνατό για διάφορους λόγους. Μπορεί ο πηγαίος κώδικας της κλάσης να μην είναι διαθέσιμος, η κλάση να έχει δηλωθεί ως sealed, η να μην υπάρχουν τα απαραίτητα δικαιώματα για να τροποποιηθεί.

Σε αυτή την περίπτωση μπορούν να χρησιμοποιηθούν extension methods τα οποία επιτρέπουν την προσθήκη μεθόδων σε υπάρχουσες κλάσεις σε κάθε περίπτωση χωρίς να επηρεάζουν τον αρχικό κώδικα και χωρίς να απαιτούν την επαναμεταγλώττισή του.

Για τη δημιουργία extension methods είναι απαραίτητη η κατασκευή μίας \codebox{static} κλάσης. Σε αυτή την κλάση μπορούν να προστεθούν static μέθοδοι, οι οποίες ονομάζονται extension methods αν πριν το πρώτο όρισμά τους τοποθετηθεί η λέξη κλειδί \codebox{this}. Έτσι, η μέθοδος γίνεται extension method της κλάσης του πρώτου της ορίσματος και μπορεί να καλείται και κάθε στιγμιότυπο αυτής της κλάσης.

Για παράδειγμα, στον κώδικα \ref{stringExtensions} η μέθοδος \codebox{CountOccuranciesOfChar} αποτελεί extension method της κλάσης \codebox{string}. Δέχεται ως πρώτο όρισμα \\
 το \codebox{this string data} που υποδηλώνει ότι είναι extension της κλάσης \codebox{string} και ως δεύτερο το \codebox{char character}, το οποίο χρησιμοποεί κατά την εκτέλεση της μεθόδου για να μετρήσει πόσες φορές εμφανίζεται στο \codebox{data} ο χαρακτήρας που ορίζεται. 

\begin{listing}[htbp]
\begin{minted}[]{csharp}
internal static class StringExtensions
{
    public static int CountOccuranciesOfChar(this string data, 
                                                char character)
    {
        return data.Count(d => d == character);
    }
}
\end{minted}
\caption{Extension μέθοδος για ένα string}
\label{stringExtensions}
\end{listing}

Στον κώδικα \ref{extensionExec} καλείται η παραπάνω μέθοδος για το string \codebox{fox}, προκειμένου να μετρηθεί το πλήθος των χαρακτήρων \codebox{'o'} και \codebox{'a'}. Παρατηρείται ότι παρότι η μέθοδος ορίζεται με δύο ορίσματα, καλείται μόνο με ένα. Αυτό συμβαίνει διότι, ως extension method, το πρώτο όρισμα το δέχεται απευθείας από τη string μεταβλητή μέσω της οποίας καλείται.

\begin{listing}[htbp]
\begin{minted}[]{csharp}
string fox = "The quick brown fox jumps over the lazy dog";

int countO = fox.CountOccuranciesOfChar('o');
int countA = fox.CountOccuranciesOfChar('a');

Console.WriteLine($"O: {countO}, A: {countA}"); // O: 4, A: 1
\end{minted}
\caption{Χρήση extension μεθόδου}
\label{extensionExec}
\end{listing}