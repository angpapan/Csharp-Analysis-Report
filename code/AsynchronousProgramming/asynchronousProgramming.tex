Ασύγχρονος προγραμματισμός υπάρχει όταν ο κώδικας ξεκινάει μια λειτουργία μακράς διάρκειας (long-running operation), αλλά δεν περιμένει για όσο διαρκεί αυτή η λειτουργία. Λειτουργίες μακράς διάρκειας θεωρούνται η πρόσβαση σε δίσκους, σε βάσεις δεδομένων, στο δίκτυο, και γενικότερα καθυστερήσεις για κάποιο χρονικό διάστημα. Σε όλες τις γνωστές γλώσσες προγραμματισμού, ο κώδικας τρέχει σε ένα thread του λειτουργικού συστήματος. Εάν όσο βρίσκεται σε εξέλιξη κάποια λειτουργία μακράς διάρκειας, αυτό το thread συνεχίζει να κάνει άλλες ενέργειες, τότε λέμε ότι έχουμε ασύγχρονο κώδικα. 

Στην C\#, ο ασύγχρονος προγραμματισμός υποστηρίζεται από τις λέξεις-κλειδιά async και await.
\begin{itemize}
    \item Η λέξη-κλειδί \codebox{async} ορίζει μία μέθοδο ως ασύγχρονη. Η μέθοδος αυτή μπορεί να έχει μία ή περισσότερες εκφράσεις await.
    \item η λέξη-κλειδί \codebox{await} βρίσκεται μέσα σε μία async μέθοδο και χρησιμοποιείται για την ασύγχρονη αναμονή ολοκλήρωσης μιας εργασίας χωρίς να μπλοκάρεται το thread κλήσης. 
\end{itemize}

Τα οφέλη του ασύγχρονου προγραμματισμού συνοψίζονται σε τρία κεντρικά σημεία: 
\begin{itemize}
    \item Βελτιωμένη ανταπόκριση \textbf{(Responsiveness)}: Τα web apps και γενικότερα οι διεπαφές χρήστη μπορούν να ανταποκρίνονται άμεσα, ενώ παράλληλα εκτελούν εργασίες μεγάλης διάρκειας στο παρασκήνιο. 
    \item Επεκτασιμότητα \textbf{(Scalability)}: Ο ασύγχρονος προγραμματισμός επιτρέπει την αποτελεσματική χρήση των πόρων του συστήματος, με αποτέλεσμα την καλύτερη διαχείριση του αυξανόμενου φόρτου εργασίας, ή της αυξανόμενης ζήτησης. Η επεκτασιμότητα μιας εφαρμογής βελτιώνεται επιτρέποντάς της να χειρίζεται περισσότερες ταυτόχρονες εργασίες ή αιτήσεις χωρίς γραμμική αύξηση της χρήσης των πόρων.
    \item Μειωμένη κατανάλωση πόρων \textbf{(Reduced Recourse Consumption)}: Οι ασύγχρονες μέθοδοι καταναλώνουν λιγότερους πόρους συστήματος σε σύγκριση με τις σύγχρονες μεθόδους, καθώς δεν δεσμεύουν νήματα περιμένοντας την ολοκλήρωση των λειτουργιών εισόδου/εξόδου. Αποτέλεσμα αυτού είναι η καλύτερη αξιοποίηση των πόρων και η βελτιωμένη απόδοση.
\end{itemize}

\begin{listing}[htbp]
\begin{minted}[]{csharp}
public async Task<string> DownloadWebPageAsync(string url)
{
    using (HttpClient client = new HttpClient())
    {
        //Asynchronously send a GET request to the specified URL
        HttpResponseMessage resp = await client.GetAsync(url);
        
        //Asynchronously read the content of the response
        string cont = await resp.Content.ReadAsStringAsync();
        
        return cont; // Return the downloaded content
    }
}
\end{minted}
\caption{Παράδειγμα χρήσης ασύγχρονης μεθόδου}
\label{asyncExample}
\end{listing}