\begin{listing}[htbp]
\begin{minted}[]{csharp}
using System;

namespace HelloWorld
{
    class Program
    {
        static void Main(string[] args)
        {
            Console.WriteLine("Hello World!");
        }
    }
}
\end{minted}
\caption{Το πρώτο πρόγραμμα σε C\#}
\label{HelloWorld}
\end{listing}

\codebox{Using System}: Η Γραμμή αυτή περιλαμβάνει τον χώρο ονομάτων (namespace) System, ο οποίος περιέχει θεμελιώδεις κλάσεις και βασικούς τύπους. Σε αυτή τη περίπτωση χρησιμοποιείται για την πρόσβαση στη κλάση Console, για λειτουργίες εισόδου - εξόδου.

\codebox{namespace HelloWord}: Τα namespaces παρουσιάζουν ομοιότητες με τα packages της Java. Είναι λογικοί containers, υπό την έννοια ότι οργανώνουν νοηματικά τις κλάσεις, χωρίς να αντιστοιχούν απαραίτητα σε φακέλους στο σύστημα.

\codebox{class Program}: Οι κλάσεις είναι containers για δεδομένα και μεθόδους, και αποτελούν θεμελιώδη δομικά στοιχεία του αντικειμενοστραφούς προγραμματισμού. ΚΑΘΕ γραμμή κώδικα που γράφεται σε C\#, πρέπει να βρίσκεται μέσα σε μια κλάση. Τη συγκεκριμένη κλάση την ονομάσαμε Program.

\codebox{static void Main}: Όπως και σε άλλες γλώσσες προγραμματισμού, η μέθοδος Main αποτελεί το σημείο εκκίνησης του προγράμματος. Με τον όρο static δηλώνεται ότι η μέθοδος ανοίκει στην ίδια την κλάση, και όχι σε κάποιο στιγμιότυπο αυτής. Με τον όρο void, η Main (και δηλαδή το πρόγραμμα) δεν επιστρέφει καμία τιμή. 

\codebox{Console.WriteLine(”Hello World!”)}: Τυπώνεται στην οθόνη η φράση "Hello World!". Η WriteLine είναι μια μέθοδος της κλάσης Console, η οποία λαμβάνει ως όρισμα ένα αλφαριθμητικό (string) και το τυπώνει στην κονσόλα, μαζί με έναν χαρακτήρα αλλαγής γραμμής.

Από το πρώτο κιόλας πρόγραμμα, μπορεί κάποιος να διακρίνει ομοιότητες με την Java: 

Η σύνταξη των δύο γλωσσών μοιάζει αρκετά, ιδίως στον τρόπο που συντάσσονται οι κλάσεις και οι μέθοδοι. Και οι δύο γλώσσες είναι αντικειμενοστραφείς, υποστηρίζοντας κλάσεις, αντικείμενα, κληρονομικότητα, πολυμορφισμό κ.α.

Η μέθοδος Main έχει παρόμοια σύνταξη και υπηρετεί τον ίδιο σκοπό. 

Standard Libraries: H C\# και η Java διαθέτουν μεγάλη ποικιλία από βιβλιοθήκες που παρέχουν ένα μεγάλο φάσμα από λειτουργίες. Στο πρόγραμμα Hello World, η "System.Console.WriteLine" της C\#, είναι η αντίστοιχη "System.out.println" της Java. 