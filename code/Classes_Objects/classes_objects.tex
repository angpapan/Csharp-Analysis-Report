Οι κλάσεις και τα αντικείμενα είναι οι δύο βασικές ιδέες του αντικειμενοστραφούς προγραμματισμού. Η διαφορά μεταξύ τους γίνεται εύκολα κατανοητή με την παρακάτω εικόνα.

\begin{table}[h]
    \centering
    \small{
    \begin{tabular}{|l|l|}
        \hline
        \textbf{Κλάση} & \textbf{Αντικείμενο} \\ \hline
        \multirow{3}{*}{Animal} & Cat \\ \cline{2-2}
        & Dog \\ \cline{2-2}
        & Snake \\ \hline
        \multirow{3}{*}{Car} & Toyota \\ \cline{2-2}
        & Hyundai \\ \cline{2-2}
        & Kia \\ \hline
    \end{tabular}
    }
    \label{tab:my_label}
\end{table}

Κλάση είναι το πρωτότυπο, η γενική ιδέα. Αντικείμενο είναι ένα στιμγιότυπο της κλάσης. Όταν ένα στιγμιότυπο μιας κλάσης δημιουργείται, αποκτά τα χαρακτηριστικά και τις μεθόδους αυτής. Τα πάντα στη C\# σχετίζονται με τις κλάσεις.

Για να δημιουργηθεί μια κλάση, χρησιμοποιείται η λέξη-κλειδή \codebox{class}. Για να δημιουργηθεί ένα αντικείμενο μιας κλάσης, χρησιμοποιείται η λέξη-κλειδί \codebox{new} ακολουθούμενο από το όνομα της κλάσης.

\begin{listing}[H]
\begin{minted}[]{csharp}
class Program
{
    internal class Animal
    {
        public int legsNum;
    }

    static void Main(string[] args)
    {
        // Creating a new instance of the Animal class
        Animal dog = new Animal();
        dog.legsNum = 4;
        
        // Another instance of the Animal class
        Animal snake = new Animal();
        snake.legsNum = 0;
        
        Console.WriteLine("Dog has {0} legs", dog.legsNum);
        Console.WriteLine("Snake has {0} legs" ,snake.legsNum);
    }
}
\end{minted}
\caption{Απλή κλάση}
\label{animal_simple}
\end{listing}