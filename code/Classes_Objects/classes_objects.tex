Οι κλάσεις και τα αντικείμενα είναι οι δύο βασικές ιδέες του αντικειμενοστραφούς προγραμματισμού. Η διαφορά μεταξύ τους γίνεται εύκολα κατανοητή με την παρακάτω εικόνα.

\begin{table}[h]
    \centering
    \small{
    \begin{tabular}{|l|l|}
        \hline
        \textbf{Κλάση} & \textbf{Αντικείμενο} \\ \hline
        \multirow{3}{*}{Animal} & Cat \\ \cline{2-2}
        & Dog \\ \cline{2-2}
        & Snake \\ \hline
        \multirow{3}{*}{Car} & Toyota \\ \cline{2-2}
        & Hyundai \\ \cline{2-2}
        & Kia \\ \hline
    \end{tabular}
    }
    \label{tab:my_label}
\end{table}

Κλάση είναι το πρωτότυπο, η γενική ιδέα. Αντικείμενο είναι ένα στιμγιότυπο της κλάσης. Όταν ένα στιγμιότυπο μιας κλάσης δημιουργείται, αποκτά τα χαρακτηριστικά και τις μεθόδους αυτής. Τα πάντα στη C\# σχετίζονται με τις κλάσεις.

Για να δημιουργηθεί μια κλάση, χρησιμοποιείται η λέξη-κλειδή \codebox{class}. Για να δημιουργηθεί ένα αντικείμενο μιας κλάσης, χρησιμοποιείται η λέξη-κλειδί \codebox{new} ακολουθούμενο από το όνομα της κλάσης.

\input{code/Classes_Objects/code/simple_class}