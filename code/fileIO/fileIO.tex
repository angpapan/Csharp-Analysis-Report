Οι λειτουργίες εισόδου και εξόδου αρχείων (I/O) είναι απαραίτητες για πολλές εφαρμογές, επιτρέποντάς τους να διαβάζουν από αρχεία και να γράφουν σε αυτά. 
Η σειριοποίηση (Serialization) επιτρέπει τη μετατροπή των αντικειμένων σε μορφή που μπορεί εύκολα να αποθηκευτεί ή να μεταδοθεί και αργότερα να ανακατασκευαστεί.

H C\# παρέχει αρκετές κλάσεις για λειτουργίες εισόδου/εξόδου αρχείων, στον χώρο ονομάτων System.IO. 
Οι κύριες κλάσεις που χρησιμοποιούνται για την είσοδο/έξοδο αρχείων είναι οι \textbf{FileStream}, \textbf{StreamReader},\textbf{StreamWriter}, \textbf{FIle} και \textbf{Directory}.

Στο παράδειγμα του κώδικα \ref{ReadFile} η κλάση \textbf{StreamReader} χρησιμοποιείται για την ανάγωνση ολόκληρουυ του περιεχομένου ενός αρχείου. 
Η δήλωση \codebox{using} εξασφαλίζει ότι το αρχείο κλείνει σωστά μετά την ανάγνωση, ακόμα και αν προκύψει κάποια εξαίρεση.

Στο παράδειγμα \ref{WriteFile}, η κλάση \textbf{StreamWriter} χρησιμοποιείται για την εγγραφή μιας συμβολοσειράς σε ένα αρχείο. Και πάλι, η δήλωση \codebox{using} εξασφαλίζει ότι το αρχείο κλείνει μετά την εγγραφή.

\input{code/fileIO/code/read}
\input{code/fileIO/code/write}

Σειριοποίση είναι η διαδικασία μετατροπής ενός αντικειμένου σε μορφή που μπορεί εύκολα να αποθηκευτεί ή να μεταδοθεί, όπως XML ή JSON.
H αποσειριοποίηση είναι η αντίστροφη διαδικασία, όπου η σειριοποιημένη μορφή μετατρέπεται ξανά σε αντικείμενο.

\input{code/fileIO/code/XML_serialization}
\input{code/fileIO/code/JSON_serialization}