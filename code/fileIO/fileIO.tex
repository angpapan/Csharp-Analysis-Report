Οι λειτουργίες εισόδου και εξόδου αρχείων (I/O) είναι απαραίτητες για πολλές εφαρμογές, επιτρέποντάς τους να διαβάζουν από αρχεία και να γράφουν σε αυτά. 
Η σειριοποίηση (Serialization) επιτρέπει τη μετατροπή των αντικειμένων σε μορφή που μπορεί εύκολα να αποθηκευτεί ή να μεταδοθεί και αργότερα να ανακατασκευαστεί.

H C\# παρέχει αρκετές κλάσεις για λειτουργίες εισόδου/εξόδου αρχείων, στον χώρο ονομάτων System.IO. 
Οι κύριες κλάσεις που χρησιμοποιούνται για την είσοδο/έξοδο αρχείων είναι οι \textbf{FileStream}, \textbf{StreamReader},\textbf{StreamWriter}, \textbf{FIle} και \textbf{Directory}.

Στο παράδειγμα του κώδικα \ref{ReadFile} η κλάση \textbf{StreamReader} χρησιμοποιείται για την ανάγωνση ολόκληρουυ του περιεχομένου ενός αρχείου. 
Η δήλωση \codebox{using} εξασφαλίζει ότι το αρχείο κλείνει σωστά μετά την ανάγνωση, ακόμα και αν προκύψει κάποια εξαίρεση.

Στο παράδειγμα \ref{WriteFile}, η κλάση \textbf{StreamWriter} χρησιμοποιείται για την εγγραφή μιας συμβολοσειράς σε ένα αρχείο. Και πάλι, η δήλωση \codebox{using} εξασφαλίζει ότι το αρχείο κλείνει μετά την εγγραφή.

\begin{listing}[htbp]
\begin{minted}[]{csharp}
using System;
using System.IO;

class FileReadExample
{
    static void Main()
    {
        string path = "example.txt";

        try
        {
            using (StreamReader sr = new StreamReader(path))
            {
                string content = sr.ReadToEnd();
                Console.WriteLine("File Contnent: " + content);
            }
        } 
        catch (Exception e)
        {
            Console.WriteLine("The file could not be read:");
            Console.WriteLine(e.Message);
        }
    }
}
\end{minted}
\caption{Διάβασμα από αρχείο}
\label{ReadFile}
\end{listing}
\begin{listing}[H]    
\begin{minted}[]{csharp}
using System;
using System.IO;

class FileWriteExample
{
    static void Main()
    {
        string path = "example.txt";
        string content = "Hello, World!";

        try
        {
            using (StreamWriter sw = new StreamWriter(path))
            {
                sw.Write(content);
            }
            Console.WriteLine("File written successfully!");
        } 
        catch (Exception e)
        {
            Console.WriteLine("Could not write to file:");
            Console.WriteLine(e.Message);
        }
    }
}
\end{minted}
\caption{Γράψιμο σε αρχείο}
\label{WriteFile}
\end{listing}


Σειριοποίση είναι η διαδικασία μετατροπής ενός αντικειμένου σε μορφή που μπορεί εύκολα να αποθηκευτεί ή να μεταδοθεί, όπως XML ή JSON.
H αποσειριοποίηση είναι η αντίστροφη διαδικασία, όπου η σειριοποιημένη μορφή μετατρέπεται ξανά σε αντικείμενο.

\begin{listing}[htbp]
\begin{minted}[]{csharp}
using System;
using System.IO;
using System.Xml.Serialization;

public class Person
on
{
    public string Name { get; set; }
    public int Age { get; set; }
}

class XmlSerializationExample
{
static void Main()
{
Person p = new Person { Name = "John Doe", Age = 30};
XmlSerializer serializer = new SmlSerializer(typeof(Person));

// Serialize object to XML
using (FileStream fs = new FileStream("person.xml", FileMode.Create))
{
    serializer.Serialize(fs, p);
    Console.WriteLine("Object serialized to XML!");
}

// Deserialize object from XML
using (FileStream fs = new FileStream("person.xml", FileMode.Open))
{
    Person p2 = (Person)serializer.Deserialize(fs);
    Console.WriteLine("Object deserialized from XML:");
    Console.WriteLine("Name: " + p2.Name);
    Console.WriteLine("Age: " + p2.Age);
}
}
}
\end{minted}
\caption{XML Serialization and Deserialization} 
\label{XMLSerialization}
\end{listing}

\begin{listing}[H]
\begin{minted}[]{csharp}
using System;
using System.IO;
using System.Text.Json;

public class Person
{
    public string Name { get; set; }
    public int Age { get; set; }
}

class JsonSerializationExample
{
static void Main()
{
Person person = new Person { Name = "John Doe", Age = 30 };

// Serialize to JSON
string jsonString = JsonSerializer.Serialize(person);
File.WriteAllText("person.json", jsonString);
Console.WriteLine("Object serialized to JSON.");

// Deserialize from JSON
jsonString = File.ReadAllText("person.json");
Person des_person=JsonSerializer.Deserialize<Person>(jsonString);
Console.WriteLine("Object deserialized from JSON:");
Console.WriteLine($"Name: {des_person.Name}");
Console.WriteLine($"Age: {des_person.Age}");
}
}
\end{minted}
\caption{JSON Serialization and Deserialization} 
\label{JSONSerialization}
\end{listing}