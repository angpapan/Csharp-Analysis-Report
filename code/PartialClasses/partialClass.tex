H C\# επιτρέπει τη διάσπαση του ορισμού μίας κλάσης σε περισσότερα αρχεία με τη χρήση της δεσμευμένης λέξης \codebox{partial} ανάμεσα στον access modifier και τη λέξη \codebox{class}. Είναι ένα ιδιαίτερα χρήσιμο χαρακτηριστικό όταν διαφορετικοί προγραμματιστές εργάζονται στην ίδια κλάση, αποφεύγοντας τις ταυτόχρονες αλλαγές στο ίδιο αρχείο.

Για τη δημιουργία μίας partial κλάσης απαιτείται σε όλα τα αρχεία η κλάση να έχει το ίδιο όνομα, την ίδια ορατότητα, να δηλώνεται σε όλα ως \codebox{partial} και να βρίσκονται στο ίδιο \codebox{namespace} ώστε κατά τον χρόνο μεταγλώττισης να σχηματίσουν έναν ενιαίο τελικό τύπο.

\begin{listing}[htbp]
\begin{minted}[]{csharp}
// Dog.cs
internal partial class Dog
{
    public string Name { get; }
    public Dog(string name) { Name = name; }
}

// DogDev1.cs
internal partial class Dog
{
    private int _distance;
    public int Distance => _distance;
    public void Walk() => _distance++;
}

// DogDev2.cs
internal partial class Dog
{
    public void Bark() => 
        Console.WriteLine($"{Name} says: Woof woof!");
}
\end{minted}
\caption{Partial κλάση σε διαφορετικά αρχεία}
\label{PartialDogClasses}
\end{listing}

Ένα παράδειγμα δημιουργίας partial κλάσης φαίνεται στον κώδικα \ref{PartialDogClasses}, ενώ στον κώδικα \ref{PartialExec} στον οποίο χρησιμοποιείται η παραπάνω κλάση διαφαίνεται πως τα διαφορετικά αρχεία συμπεριφέρονται σαν μία κλάση.

\begin{listing}[htbp]
\begin{minted}[]{csharp}
Dog doggo = new Dog("Max");

doggo.Walk();
doggo.Bark();
doggo.Walk();

Console.WriteLine(
    $"{doggo.Name} walked {doggo.Distance} steps!");
\end{minted}
\caption{Χρήση Partial κλάσης}
\label{PartialExec}
\end{listing}

Κάποια επιπλέον ιδιαίτερα χαρακτηριστικά που αφορούν τις partial κλάσεις είναι τα εξής:
\begin{itemize}
    \item Αν η κλάση σε οποιοδήποτε αρχείο χαρακτηριστεί ως abstract ή sealed ολόκληρη η κλάση θεωρείται τέτοια.
    \item Αν η κλάση κληρονομεί από κάποια άλλη σε κάποιο αρχείο, τότε την κληρονομεί σε όλα.
    \item Κάθε πεδίο που ορίζεται σε κάποιο αρχείο είναι διαθέσιμο και σε όλα τα άλλα.
\end{itemize}
\newpage