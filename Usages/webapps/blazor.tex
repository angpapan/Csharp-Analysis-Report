Το Blazor αποτελεί ένα framework για την ανάπτυξη διαδραστικών διεπαφών με τη χρήση C\# αντί για javascript,
, με δύο εκδοχές:
\begin{itemize}
    \item \textbf{Blazor Server,} με την οποία εκτελείται στον διακομιστή, με τις ενημερώσεις της διεπαφής να
            αποστέλλονται στον client μέσω μίας σύνδεσης SignalR.
    \item \textbf{Blazor WebAssembly,} με την οποία εκτελείται απευθείας στον φυλλομετρητή ιστού με χρήση
            WebAssembly. 
\end{itemize}

Με τη χρήση του Blazor ολόκληρη η εφαρμογή δύναται να χρησιμοποιεί αποκλειστικά τη γλώσσα C\#, τόσο στον
server όσο και στον client. Έτσι, περιορίζεται η βάση κώδικα και οι απαιτήσεις γνώσης διαφορετικών γλωσσών 
από τους προγραμματιστές και διευκολύνεται ο διαμοιρασμός και η επαναχρησιμοποίηση του κώδικα.

Η Blazor χρησιμοποιεί μία component-based αρχιτεκρονική, παρόμοια με αυτή δημοφιλών framework της javascript
όπως οι React, Angular και Vue.