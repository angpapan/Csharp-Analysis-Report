To ASP.NET MVC (Model-View-Controller) αποτελεί ένα framework για την ανάπτυξη διαδικτυακών εφαρμογών
με δομημένο τρόπο διαχωρίζοντας την εφαρμογή σε τρία κύρια στοιχεία:
\begin{itemize}
    \item \textbf{Model:} Αναπαριστά τα δεδομένα της εφαρμογής και την επιχειρησιακή της λογική. Τα μοντέλα
            είναι υπεύθυνα για την ανάκτηση και την χρήση των δεδομένων από τη βάση δεδομένων.
     \item  \textbf{View:} Αποτελεί το περιβάλλον χρήσης της εφαρμογής. Είναι υπεύθυνο για την προβολή
            των δεδομένων στον χρήστη και τις σχετικές επιλογές του. Συνήθως για την κατασκευή τους
            χρησιμοποιούνται HTML, CSS και στοιχεία Razor.
    \item  \textbf{Controller:} Χειρίζεται τις αλληλεπιδράσεις και τις καταχωρήσεις των χρηστών. 
            Επεξεργάζεται τα εισερχόμενα αιτήματα, εκτελεί εργασίες πάνω στο μοντέλο και επιλέγει
            τα views που θα προβληθούν στους χρήστες.
\end{itemize}

Διαχωρίζοντας την εφαρμογή σε models, view και controllers υποστηρίζει την ανάπτυξή της με καθαρή 
αρχιτεκτονική (clean architecture), διευκολύνοντας την διαχείριση, την κλιμάκωση και την συγγραφή
unit tests.