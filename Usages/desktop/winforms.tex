Τα Windows Forms είναι από τα πιο παλιά και γνωστά framework για την ανάπτυξη desktop εφαρμογών με C\#. 
Είναι ένα αρκετά απλό και εύχρηστο framework, κάνοντάς το ιδανική επιλογή για αρχάριους.
Παρέχει δυνατότητα κατασκευής του γραφικού περιβάλλοντος της εφαρμογής μέσω του drag-and-drop designer 
του Visual Studio χρησιμοποιώντας μία πληθώρα πλήκτρων ελέγχου (κουμπιά, checkboxes  και άλλα) τα οποία 
μπορούν να προστεθούν στις φόρμες. Ως βασική μέθοδος προγραμματισμού χρησιμοποιείται ο προγραμματισμός
που βασίζεται στον χειρισμό γεγονότων (event driven programming), επιτρέποντας τον άμεσο χειρισμό των
ενεργειών των χρηστών. 

Η ανάπτυξη της εφαρμογής βασίζεται σε φόρμες, από τις οποίες αποτελείται κάθε παράθυρο και διάλογος.
Η επιχειρησιακή λογική αναπτύσσεται στα αντίστοιχα code-behind αρχεία, ενώ μέθοδοι συνδεδεμένες με 
γεγονότα ελέγχου (για παράδειγμα κλικ σε κουμπιά ή είσοδος κειμένου) χειρίζονται τις ενέργειες των χρηστών.
