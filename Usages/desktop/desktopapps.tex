Ακόμη μία βασική χρήση της C\# είναι για την ανάπτυξη desktop εφαρμογών,
δηλαδή εφαρμογών που εγκαθίστανται στον υπολογιστή του χρήστη και λειτουργούν
απευθείας από αυτόν χωρίς να χρειάζεται σύνδεση στο διαδίκτυο. Όπως και οι διαδικτυακές
εφαρμογές, έτσι και οι desktop βασίζονται στο .NET. Τα κύρια framework που χρησιμοποιούνται 
για την ανάπτυξή τους είναι τα:
\begin{itemize}
    \item Windows Forms (WinForms)
    \item Windows Presentation Foundation (WPF)
    \item MAUI
\end{itemize}
Όλα τα παραπάνω frameworks αποτελούν προϊόντα της Microsoft.

\subsubsection{Windows Forms (WinForms)}
Τα Windows Forms είναι από τα πιο παλιά και γνωστά framework για την ανάπτυξη desktop εφαρμογών με C\#. 
Είναι ένα αρκετά απλό και εύχρηστο framework, κάνοντάς το ιδανική επιλογή για αρχάριους.
Παρέχει δυνατότητα κατασκευής του γραφικού περιβάλλοντος της εφαρμογής μέσω του drag-and-drop designer 
του Visual Studio χρησιμοποιώντας μία πληθώρα πλήκτρων ελέγχου (κουμπιά, checkboxes  και άλλα) τα οποία 
μπορούν να προστεθούν στις φόρμες. Ως βασική μέθοδος προγραμματισμού χρησιμοποιείται ο προγραμματισμός
που βασίζεται στον χειρισμό γεγονότων (event driven programming), επιτρέποντας τον άμεσο χειρισμό των
ενεργειών των χρηστών. 

Η ανάπτυξη της εφαρμογής βασίζεται σε φόρμες, από τις οποίες αποτελείται κάθε παράθυρο και διάλογος.
Η επιχειρησιακή λογική αναπτύσσεται στα αντίστοιχα code-behind αρχεία, ενώ μέθοδοι συνδεδεμένες με 
γεγονότα ελέγχου (για παράδειγμα κλικ σε κουμπιά ή είσοδος κειμένου) χειρίζονται τις ενέργειες των χρηστών.


\subsubsection{Windows Presentation Foundation (WPF)}
Το WPF αποτελεί ένα πιο σύγχρονο framework από το WinForms. Παρέχει μία πλουσιότερη εμπειρία χρήσης
με προηγμένες δυνατότητες γραφικών και σχεδίασης.

Με το WPF η διεπαφή χρήση σχεδιάζεται με τη δηλωτική γλώσσα XAML, διαχωρίζοντας έτσι το γραφικό 
περιβάλλον από την επιχειρησιακή λογική. Επιπλέον, προσφέρονται ισχυρές δυνατότητες σύνδεσης 
δεδομένων διευκολύνοντας τη σύνδεση των στοιχείου της διεπαφής χρήστη με τις πηγές δεδομένων.
Τέλος, διατίθεται πληθώρα από πρότυπα τα οποία δύνανται να τροποποιηθούν στις απαιτήσεις της
εκάστοτε περίπτωσης, ενώ μπορεί να χειρίζεται κάθε τύπο γραφικών, κινούμενων εικόνων και πολυμέσων
συμβάλλοντας στη δημιουργία πλούσιων αισθητικά εφαρμογών.

Συνήθως οι εφαρμογές που αναπτύσσονται με το WPF χρησιμοποιούν την MVVM (Model-View-ViewModel) 
αρχιτεκτονική, όπου τα μοντέλα (Model) αναπαριστούν τα δεδομένα και την επιχειρησιακή λογική,
τα Views την διεπαφή χρήστη με χρήση XAML και τα ViewModels λειτουργούν ως ενδιάμεσοι των μοντέλων
και των Views συνδέοντας τα δεδομένα με τη διεπαφή χρήστη.


\subsubsection{.ΝΕΤ MAUI (Multi-platform App UI)}
Το .NET MAUI αποτελεί εξέλιξη ενός προηγούμενου framework, του Xamarin.Forms, και αποσκοπεί στην ανάπτυξη
cross-platform desktop και mobile εφαρμογών με κοινή βάση κώδικα για Windows, maxOS, iOS και Android.
Όπως και το WPF, χρησιμοποιεί τη δηλωτική γλώσσα XAML για την κατασκευή της διεπαφής χρήστη. Ως 
αρχιτεκτονικές ανάπτυξης χρησιμοποιούνται τόσο η MVVM, η οποία αναφέρθηκε και στο WPF, αλλά και η MVU
(Model - View - Update). Στην τελευταία τα μοντέλα (Model) αναπαριστούν την κατάσταση της εφαρμογής,
τα Views τη διεπαφή χρήστη, ενώ τα Updates χειρίζονται τις αλλαγές της κατάστασης της εφαρμογής 
ενημερώνοντας ταυτόχρονα και το περιβάλλον χρήσης.