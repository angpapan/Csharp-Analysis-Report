Το WPF αποτελεί ένα πιο σύγχρονο framework από το WinForms. Παρέχει μία πλουσιότερη εμπειρία χρήσης
με προηγμένες δυνατότητες γραφικών και σχεδίασης.

Με το WPF η διεπαφή χρήση σχεδιάζεται με τη δηλωτική γλώσσα XAML, διαχωρίζοντας έτσι το γραφικό 
περιβάλλον από την επιχειρησιακή λογική. Επιπλέον, προσφέρονται ισχυρές δυνατότητες σύνδεσης 
δεδομένων διευκολύνοντας τη σύνδεση των στοιχείου της διεπαφής χρήστη με τις πηγές δεδομένων.
Τέλος, διατίθεται πληθώρα από πρότυπα τα οποία δύνανται να τροποποιηθούν στις απαιτήσεις της
εκάστοτε περίπτωσης, ενώ μπορεί να χειρίζεται κάθε τύπο γραφικών, κινούμενων εικόνων και πολυμέσων
συμβάλλοντας στη δημιουργία πλούσιων αισθητικά εφαρμογών.

Συνήθως οι εφαρμογές που αναπτύσσονται με το WPF χρησιμοποιούν την MVVM (Model-View-ViewModel) 
αρχιτεκτονική, όπου τα μοντέλα (Model) αναπαριστούν τα δεδομένα και την επιχειρησιακή λογική,
τα Views την διεπαφή χρήστη με χρήση XAML και τα ViewModels λειτουργούν ως ενδιάμεσοι των μοντέλων
και των Views συνδέοντας τα δεδομένα με τη διεπαφή χρήστη.
