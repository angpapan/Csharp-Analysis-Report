τα τελευταία χρόνια η C\# έχει γίνει αρκετά δημοφιλής και για την ανάπτυξη παιχνιδιών. Η ανάπτυξη 
μηχανών (game engines) γύρω από τη γλώσσα συνέβαλλε καθοριστικά στην είσοδό της σε αυτή τη βιομηχανία.
 
Η Unity είναι αναμφισβήτητα η πιο δημοφιλής μηχανή παιχνιδιών που χρησιμοποιεί τη C\#. Προσφέρει μία 
ολοκληρωμένη σειρά εργαλείων για τη δημιουργία δισδιάστατων και τρισδιάστατων παιχνιδιών. Η αρχιτεκτονική 
της Unity που βασίζεται σε components μαζί με την εκτεταμένη βιβλιοθήκη έτοιμων στοιχείων διευκολύνει
τους προγραμματιστές στην δημιουργία παιχνιδιών με πλούσια εμπειρία χρήστη. Επιπλέον παρέχει έναν ισχυρό
επεξεργαστή περιβάλλοντος για την κατασκευή του κόσμου του παιχνιδιού και των γραφικών και συμβάλλει στην
αποσφαλμάτωση και τη μέτρηση των επιδόσεων του κώδικα. Τέλος η Unity επιτρέπει την ανάπτυξη cross-platform 
παιχνιδιών στοχεύοντας στις πλατφόρμες Windows, maxOS, Linux, iOS, Android, φυλλομετρητές ιστού και κονσόλες.
Επιπλέον, δύνανται να κατασκευαστούν και παιχνίδια εικονικής ή επαυξημένης πραγματικότητας.

Άλλες μχηανές ανάπτυξης παιχνιδιών που βασίζονται στη C\# είναι η Godot και η MonoGame, 
οι οποίες είναι ανοιχτού κώδικα, αλλά κατέχουν πολύ μικρότερο μερίδιο αγοράς.
